% Template LaTeX pour le rapport de stage ETER
\documentclass[12pt,a4paper]{report}

% Packages essentiels
\usepackage[utf8]{inputenc}
\usepackage[T1]{fontenc}
\usepackage[french]{babel}
\usepackage{geometry}
\usepackage{fancyhdr}
\usepackage{graphicx}
\usepackage{xcolor}
\usepackage{hyperref}
\usepackage{titlesec}
\usepackage{tocloft}
\usepackage{listings}
\usepackage{float}
\usepackage{enumitem}
\usepackage{array}
\usepackage{longtable}
\usepackage{booktabs}
\usepackage{multirow}
\usepackage{amsmath}
\usepackage{amsfonts}
\usepackage{amssymb}

% Configuration de la page
\geometry{
    left=2.5cm,
    right=2.5cm,
    top=2.5cm,
    bottom=2.5cm,
    headheight=15pt
}

% Configuration des couleurs
\definecolor{eterblue}{RGB}{26, 86, 219}
\definecolor{etergray}{RGB}{108, 117, 125}
\definecolor{codegreen}{RGB}{0, 128, 0}
\definecolor{codegray}{RGB}{128, 128, 128}
\definecolor{codepurple}{RGB}{128, 0, 128}
\definecolor{backcolour}{RGB}{248, 249, 250}

% Configuration des liens
\hypersetup{
    colorlinks=true,
    linkcolor=eterblue,
    filecolor=eterblue,
    urlcolor=eterblue,
    citecolor=eterblue,
    pdftitle={Rapport de Stage S4 - ETER PWA},
    pdfauthor={Votre Nom},
    pdfsubject={Développement PWA pour rapports journaliers},
    pdfkeywords={PWA, Node.js, MongoDB, Offline, ETER}
}

% Configuration des en-têtes et pieds de page
\pagestyle{fancy}
\fancyhf{}
\fancyhead[L]{\textcolor{etergray}{\small Rapport de Stage S4}}
\fancyhead[R]{\textcolor{etergray}{\small ETER - PWA}}
\fancyfoot[C]{\textcolor{etergray}{\thepage}}
\renewcommand{\headrulewidth}{0.4pt}
\renewcommand{\footrulewidth}{0pt}

% Configuration des titres
\titleformat{\chapter}[display]
{\normalfont\huge\bfseries\color{eterblue}}
{\chaptertitlename\ \thechapter}{20pt}{\Huge}

\titleformat{\section}
{\normalfont\Large\bfseries\color{eterblue}}
{\thesection}{1em}{}

\titleformat{\subsection}
{\normalfont\large\bfseries\color{eterblue}}
{\thesubsection}{1em}{}

\titleformat{\subsubsection}
{\normalfont\normalsize\bfseries\color{eterblue}}
{\thesubsubsection}{1em}{}

% Configuration des listings de code
\lstdefinestyle{codestyle}{
    backgroundcolor=\color{backcolour},
    commentstyle=\color{codegreen},
    keywordstyle=\color{eterblue},
    numberstyle=\tiny\color{codegray},
    stringstyle=\color{codepurple},
    basicstyle=\ttfamily\footnotesize,
    breakatwhitespace=false,
    breaklines=true,
    captionpos=b,
    keepspaces=true,
    numbers=left,
    numbersep=5pt,
    showspaces=false,
    showstringspaces=false,
    showtabs=false,
    tabsize=2,
    frame=single,
    rulecolor=\color{etergray}
}

\lstset{style=codestyle}

% Configuration des listes
\setlist[itemize]{label=\textcolor{eterblue}{\textbullet}}
\setlist[enumerate]{label=\textcolor{eterblue}{\arabic*.}}

% Configuration de la table des matières
\renewcommand{\cftchapfont}{\bfseries\color{eterblue}}
\renewcommand{\cftsecfont}{\color{eterblue}}
\renewcommand{\cftsubsecfont}{\color{etergray}}

% Commandes personnalisées
\newcommand{\tech}[1]{\textcolor{eterblue}{\textbf{#1}}}
\newcommand{\code}[1]{\texttt{\textcolor{codepurple}{#1}}}
\newcommand{\file}[1]{\texttt{\textcolor{codegreen}{#1}}}

% Environnement pour les encadrés
\usepackage{mdframed}
\newmdenv[
    backgroundcolor=backcolour,
    linecolor=eterblue,
    linewidth=2pt,
    leftmargin=10pt,
    rightmargin=10pt,
    topline=true,
    bottomline=true,
    leftline=true,
    rightline=true
]{infobox}

% Début du document
\begin{document}

% Page de titre
\begin{titlepage}
    \centering
    \vspace*{2cm}
    
    % Logo université (à remplacer par votre logo)
    % \includegraphics[width=0.3\textwidth]{logo_universite.png}
    
    \vspace{1cm}
    
    {\huge\bfseries\color{eterblue} Rapport de Stage S4}
    
    \vspace{0.5cm}
    
    {\Large\bfseries Développement d'une Application PWA pour la Gestion des Rapports Journaliers ETER}
    
    \vspace{1.5cm}
    
    {\large\bfseries Digitalisation avec Approche Offline-First et Synchronisation Intelligente}
    
    \vspace{2cm}
    
    \begin{tabular}{rl}
        \textbf{Étudiant :} & [Votre Nom] \\
        \textbf{Filière :} & [Votre Filière] \\
        \textbf{Semestre :} & S4 \\
        \textbf{Année universitaire :} & [Année] \\
        \textbf{Période de stage :} & [Dates] \\
        \textbf{Encadrant académique :} & [Nom] \\
        \textbf{Encadrant professionnel :} & [Nom] \\
    \end{tabular}
    
    \vspace{2cm}
    
    {\large\bfseries Institution d'accueil}
    
    \vspace{0.5cm}
    
    {\Large\color{eterblue} ETER}
    
    {\large Établissement des Travaux d'Entretien Routier}
    
    {\normalsize Direction des Approvisionnements et Logistique}
    
    \vfill
    
    {\large \today}
\end{titlepage}

% Pages préliminaires
\pagenumbering{roman}

% Résumé
\chapter*{Résumé}
\addcontentsline{toc}{chapter}{Résumé}

Ce rapport présente le développement d'une \tech{Progressive Web App (PWA)} pour la digitalisation des rapports journaliers de l'\tech{ETER} (Établissement des Travaux d'Entretien Routier). La solution implémente une approche \tech{offline-first} avec synchronisation automatique, permettant aux employés de saisir des données même sans connexion internet.

Le projet utilise \tech{Node.js}, \tech{MongoDB}, et des technologies PWA modernes pour créer une application robuste et performante. L'architecture développée comprend une interface employé PWA, un dashboard administrateur, et un système de synchronisation intelligent capable de gérer les environnements à connectivité intermittente.

Les résultats montrent une réduction de 90\% du temps de saisie, l'élimination complète des pertes de données, et une amélioration significative de la traçabilité opérationnelle. Le code source développé représente plus de 3500 lignes, incluant des tests unitaires et une documentation complète.

\textbf{Mots-clés :} PWA, Offline-First, Node.js, MongoDB, Synchronisation, Digitalisation, ETER, Service Worker, IndexedDB

\newpage

% Remerciements
\chapter*{Remerciements}
\addcontentsline{toc}{chapter}{Remerciements}

Je tiens à exprimer ma gratitude envers toutes les personnes qui ont contribué à la réussite de ce stage :

\begin{itemize}
    \item Mon encadrant académique [Nom] pour son suivi pédagogique et ses conseils précieux
    \item Mon encadrant professionnel [Nom] pour son accompagnement technique et sa disponibilité
    \item L'équipe ETER pour leur accueil et leur collaboration
    \item Les employés qui ont testé l'application et fourni des retours constructifs
    \item L'équipe enseignante pour la formation technique qui a rendu ce projet possible
\end{itemize}

Ce stage a été une expérience enrichissante qui m'a permis d'appliquer mes connaissances théoriques dans un contexte professionnel réel.

\newpage

% Table des matières
\tableofcontents

% Liste des figures
\listoffigures

% Liste des tableaux
\listoftables

\newpage

% Corps du document
\pagenumbering{arabic}

% Le contenu sera inséré ici depuis le fichier Markdown converti
% \textbackslash{}chapter\textbackslash{}{Rapport de Stage S4 - Développement d'une Application PWA pour la Gestion des Rapports Journaliers ETER\textbackslash{}}

\textbackslash{}textbf\textbackslash{}{Étudiant :\textbackslash{}} [Votre Nom]  
\textbackslash{}textbf\textbackslash{}{Filière :\textbackslash{}} [Votre Filière]  
\textbackslash{}textbf\textbackslash{}{Semestre :\textbackslash{}} S4  
\textbackslash{}textbf\textbackslash{}{Période de stage :\textbackslash{}} [Dates]  
\textbackslash{}textbf\textbackslash{}{Encadrant académique :\textbackslash{}} [Nom]  
\textbackslash{}textbf\textbackslash{}{Encadrant professionnel :\textbackslash{}} [Nom]  
\textbackslash{}textbf\textbackslash{}{Institution d'accueil :\textbackslash{}} ETER - Établissement des Travaux d'Entretien Routier

---

\textbackslash{}section\textbackslash{}{Résumé\textbackslash{}}

Ce rapport présente le développement d'une Progressive Web App (PWA) pour la digitalisation des rapports journaliers de l'ETER. La solution implémente une approche offline-first avec synchronisation automatique, permettant aux employés de saisir des données même sans connexion internet. Le projet utilise Node.js, MongoDB, et des technologies PWA modernes pour créer une application robuste et performante.

\textbackslash{}textbf\textbackslash{}{Mots-clés :\textbackslash{}} PWA, Offline-First, Node.js, MongoDB, Synchronisation, Digitalisation, ETER

---

\textbackslash{}section\textbackslash{}{Table des Matières\textbackslash{}}

\textbackslash{}item \textbackslash{}href\textbackslash{}{\textbackslash{}#1-introduction\textbackslash{}}\textbackslash{}{Introduction\textbackslash{}}
\textbackslash{}end\textbackslash{}{itemize\textbackslash{}}
\textbackslash{}begin\textbackslash{}{itemize\textbackslash{}}
\textbackslash{}item \textbackslash{}href\textbackslash{}{\textbackslash{}#2-contexte-et-problématique\textbackslash{}}\textbackslash{}{Contexte et Problématique\textbackslash{}}
\textbackslash{}item \textbackslash{}href\textbackslash{}{\textbackslash{}#3-solution-proposée\textbackslash{}}\textbackslash{}{Solution Proposée\textbackslash{}}
\textbackslash{}end\textbackslash{}{itemize\textbackslash{}}
\textbackslash{}begin\textbackslash{}{itemize\textbackslash{}}
\textbackslash{}item \textbackslash{}href\textbackslash{}{\textbackslash{}#4-architecture-technique\textbackslash{}}\textbackslash{}{Architecture Technique\textbackslash{}}
\textbackslash{}item \textbackslash{}href\textbackslash{}{\textbackslash{}#5-implémentation\textbackslash{}}\textbackslash{}{Implémentation\textbackslash{}}
\textbackslash{}end\textbackslash{}{itemize\textbackslash{}}
\textbackslash{}begin\textbackslash{}{itemize\textbackslash{}}
\textbackslash{}item \textbackslash{}href\textbackslash{}{\textbackslash{}#6-défis-techniques-et-solutions\textbackslash{}}\textbackslash{}{Défis Techniques et Solutions\textbackslash{}}
\textbackslash{}item \textbackslash{}href\textbackslash{}{\textbackslash{}#7-résultats-et-évaluation\textbackslash{}}\textbackslash{}{Résultats et Évaluation\textbackslash{}}
\textbackslash{}end\textbackslash{}{itemize\textbackslash{}}
\textbackslash{}begin\textbackslash{}{itemize\textbackslash{}}
\textbackslash{}item \textbackslash{}href\textbackslash{}{\textbackslash{}#8-conclusion-et-perspectives\textbackslash{}}\textbackslash{}{Conclusion et Perspectives\textbackslash{}}
\textbackslash{}end\textbackslash{}{itemize\textbackslash{}}

---

\textbackslash{}section\textbackslash{}{1. Introduction\textbackslash{}}

\textbackslash{}subsection\textbackslash{}{1.1 Contexte du Stage\textbackslash{}}

Ce stage de fin de semestre S4 s'est déroulé dans le cadre du développement d'une solution digitale pour l'ETER (Établissement des Travaux d'Entretien Routier), une institution gouvernementale responsable de la maintenance des infrastructures routières.

\textbackslash{}subsection\textbackslash{}{1.2 Objectifs du Stage\textbackslash{}}

Les objectifs principaux de ce stage étaient :
- Analyser les processus métier existants
- Concevoir une solution digitale moderne
- Développer une application PWA complète
- Implémenter un système de synchronisation offline
- Assurer la sécurité et la qualité du code

\textbackslash{}subsection\textbackslash{}{1.3 Présentation de l'ETER\textbackslash{}}

L'ETER est un établissement public spécialisé dans l'entretien des routes. Sa Direction des Approvisionnements et Logistique (DAL) gère la distribution de carburant aux véhicules des différents chantiers. Cette activité critique nécessite un suivi précis et quotidien des consommations.

---

\textbackslash{}section\textbackslash{}{2. Contexte et Problématique\textbackslash{}}

\textbackslash{}subsection\textbackslash{}{2.1 Situation Initiale\textbackslash{}}

Avant le projet, l'ETER utilisait un système de gestion papier pour les rapports journaliers. Ce processus présentait plusieurs inconvénients majeurs :

- \textbackslash{}textbf\textbackslash{}{Processus manuel\textbackslash{}} : Saisie manuscrite des données sur formulaires papier
- \textbackslash{}textbf\textbackslash{}{Risque de perte\textbackslash{}} : Vulnérabilité aux pertes et détériorations
- \textbackslash{}textbf\textbackslash{}{Manque de traçabilité\textbackslash{}} : Difficultés de suivi et d'analyse
- \textbackslash{}textbf\textbackslash{}{Inefficacité\textbackslash{}} : Processus lent de collecte et compilation
- \textbackslash{}textbf\textbackslash{}{Problèmes de connectivité\textbackslash{}} : Sites distants avec accès internet limité

\textbackslash{}subsection\textbackslash{}{2.2 Analyse des Besoins\textbackslash{}}

L'analyse des besoins a révélé plusieurs exigences critiques :

\textbackslash{}subsubsection\textbackslash{}{2.2.1 Besoins Fonctionnels\textbackslash{}}
- Digitalisation complète du formulaire papier
- Gestion des signatures numériques
- Génération de rapports PDF fidèles
- Interface d'administration centralisée
- Système d'export et d'analyse

\textbackslash{}subsubsection\textbackslash{}{2.2.2 Besoins Non-Fonctionnels\textbackslash{}}
- \textbackslash{}textbf\textbackslash{}{Disponibilité\textbackslash{}} : Fonctionnement offline obligatoire
- \textbackslash{}textbf\textbackslash{}{Performance\textbackslash{}} : Chargement rapide et interface réactive
- \textbackslash{}textbf\textbackslash{}{Sécurité\textbackslash{}} : Protection des données sensibles
- \textbackslash{}textbf\textbackslash{}{Compatibilité\textbackslash{}} : Support multi-plateforme
- \textbackslash{}textbf\textbackslash{}{Maintenabilité\textbackslash{}} : Code structuré et documenté

\textbackslash{}subsection\textbackslash{}{2.3 Contraintes Techniques\textbackslash{}}

Les principales contraintes identifiées étaient :
- Connectivité internet intermittente sur les chantiers
- Nécessité de respecter le format papier existant
- Intégration avec les processus métier actuels
- Sécurité des données gouvernementales

---

\textbackslash{}section\textbackslash{}{3. Solution Proposée\textbackslash{}}

\textbackslash{}subsection\textbackslash{}{3.1 Vision Générale\textbackslash{}}

La solution proposée consiste en une Progressive Web App (PWA) avec architecture offline-first, permettant :
- Saisie continue des données même sans connexion
- Synchronisation automatique dès reconnexion
- Interface moderne et intuitive
- Sécurité renforcée des données

\textbackslash{}subsection\textbackslash{}{3.2 Choix Technologiques\textbackslash{}}

\textbackslash{}subsubsection\textbackslash{}{3.2.1 Stack Backend\textbackslash{}}
- \textbackslash{}textbf\textbackslash{}{Node.js 16+\textbackslash{}} : Runtime JavaScript performant
- \textbackslash{}textbf\textbackslash{}{Express.js 4.18\textbackslash{}} : Framework web minimaliste
- \textbackslash{}textbf\textbackslash{}{MongoDB 7.0\textbackslash{}} : Base de données NoSQL flexible
- \textbackslash{}textbf\textbackslash{}{JWT\textbackslash{}} : Authentification sécurisée
- \textbackslash{}textbf\textbackslash{}{PDFKit\textbackslash{}} : Génération de documents PDF

\textbackslash{}subsubsection\textbackslash{}{3.2.2 Stack Frontend\textbackslash{}}
- \textbackslash{}textbf\textbackslash{}{HTML5/CSS3\textbackslash{}} : Structure et présentation modernes
- \textbackslash{}textbf\textbackslash{}{JavaScript ES6+\textbackslash{}} : Logique client avancée
- \textbackslash{}textbf\textbackslash{}{Bootstrap 5.3\textbackslash{}} : Framework UI responsive
- \textbackslash{}textbf\textbackslash{}{Service Worker\textbackslash{}} : Capacités offline
- \textbackslash{}textbf\textbackslash{}{IndexedDB\textbackslash{}} : Stockage local persistant

\textbackslash{}subsubsection\textbackslash{}{3.2.3 Justification des Choix\textbackslash{}}

| Technologie \textbackslash{}begin\textbackslash{}{longtable\textbackslash{}}\textbackslash{}{|l|\textbackslash{}}
\textbackslash{}hline
\textbackslash{}textbf\textbackslash{}{Justification\textbackslash{}} \textbackslash{}\textbackslash{}
\textbackslash{}hline
\textbackslash{}end\textbackslash{}{longtable\textbackslash{}}
| Node.js | Écosystème riche, performance, JavaScript unifié |
| MongoDB | Flexibilité des schémas, scalabilité |
| PWA | Expérience native, fonctionnement offline |
| JWT | Stateless, sécurisé, standard |
| IndexedDB | Stockage local robuste, capacité importante |

\textbackslash{}subsection\textbackslash{}{3.3 Architecture Globale\textbackslash{}}

L'architecture suit un pattern 3-tiers moderne :

\textbackslash{}begin\textbackslash{}{lstlisting\textbackslash{}}[language=]
┌─────────────────────────────────────────┐
│           COUCHE PRÉSENTATION           │
├─────────────────────────────────────────┤
│ • Interface Employé (PWA)               │
│ • Interface Admin (Dashboard)           │
│ • Service Worker                        │
└─────────────────────────────────────────┘
                    │
┌─────────────────────────────────────────┐
│           COUCHE LOGIQUE                │
├─────────────────────────────────────────┤
│ • Serveur Node.js/Express               │
│ • API REST                              │
│ • Authentification JWT                  │
│ • Génération PDF                        │
└─────────────────────────────────────────┘
                    │
┌─────────────────────────────────────────┐
│           COUCHE DONNÉES                │
├─────────────────────────────────────────┤
│ • MongoDB (Principal)                   │
│ • IndexedDB (Offline)                   │
│ • Système de fichiers                   │
└─────────────────────────────────────────┘
\textbackslash{}end\textbackslash{}{lstlisting\textbackslash{}}

---

\textbackslash{}section\textbackslash{}{4. Architecture Technique\textbackslash{}}

\textbackslash{}subsection\textbackslash{}{4.1 Architecture PWA\textbackslash{}}

\textbackslash{}subsubsection\textbackslash{}{4.1.1 Composants PWA Implémentés\textbackslash{}}

La solution respecte les standards PWA avec :
- \textbackslash{}textbf\textbackslash{}{Web App Manifest\textbackslash{}} : Configuration d'installation
- \textbackslash{}textbf\textbackslash{}{Service Worker\textbackslash{}} : Stratégies de cache et sync
- \textbackslash{}textbf\textbackslash{}{HTTPS\textbackslash{}} : Sécurité requise
- \textbackslash{}textbf\textbackslash{}{Responsive Design\textbackslash{}} : Adaptation multi-écrans
- \textbackslash{}textbf\textbackslash{}{App Shell\textbackslash{}} : Chargement rapide

\textbackslash{}subsubsection\textbackslash{}{4.1.2 Manifest Configuration\textbackslash{}}

\textbackslash{}begin\textbackslash{}{lstlisting\textbackslash{}}[language=json]
\textbackslash{}{
  "name": "ETER Reports - Rapports Journaliers",
  "short\textbackslash{}_name": "ETER",
  "theme\textbackslash{}_color": "\textbackslash{}#1a56db",
  "background\textbackslash{}_color": "\textbackslash{}#ffffff",
  "display": "standalone",
  "orientation": "portrait",
  "start\textbackslash{}_url": "/",
  "scope": "/",
  "icons": [
    \textbackslash{}{
      "src": "/assets/icon-192x192.png",
      "sizes": "192x192",
      "type": "image/png"
    \textbackslash{}}
  ]
\textbackslash{}}
\textbackslash{}end\textbackslash{}{lstlisting\textbackslash{}}

\textbackslash{}subsection\textbackslash{}{4.2 Service Worker et Cache\textbackslash{}}

\textbackslash{}subsubsection\textbackslash{}{4.2.1 Stratégies de Cache\textbackslash{}}

Trois stratégies de cache sont implémentées :

\textbackslash{}item \textbackslash{}textbf\textbackslash{}{Cache First\textbackslash{}} : Ressources statiques (CSS, JS, images)
\textbackslash{}end\textbackslash{}{itemize\textbackslash{}}
\textbackslash{}begin\textbackslash{}{itemize\textbackslash{}}
\textbackslash{}item \textbackslash{}textbf\textbackslash{}{Network First\textbackslash{}} : Données dynamiques (API)
\textbackslash{}item \textbackslash{}textbf\textbackslash{}{Stale While Revalidate\textbackslash{}} : Équilibre performance/fraîcheur
\textbackslash{}end\textbackslash{}{itemize\textbackslash{}}

\textbackslash{}subsubsection\textbackslash{}{4.2.2 Gestion Multi-Cache\textbackslash{}}

\textbackslash{}begin\textbackslash{}{lstlisting\textbackslash{}}[language=javascript]
const CACHE\textbackslash{}_NAMES = \textbackslash{}{
    static: 'eter-static-v3.3',
    dynamic: 'eter-dynamic-v3.3',
    api: 'eter-api-v3.3'
\textbackslash{}};
\textbackslash{}end\textbackslash{}{lstlisting\textbackslash{}}

\textbackslash{}subsection\textbackslash{}{4.3 Système de Synchronisation Offline\textbackslash{}}

\textbackslash{}subsubsection\textbackslash{}{4.3.1 Architecture de Synchronisation\textbackslash{}}

Le système de synchronisation utilise une approche multi-niveaux :

\textbackslash{}item \textbackslash{}textbf\textbackslash{}{Détection réseau\textbackslash{}} : Monitoring continu du statut
\textbackslash{}end\textbackslash{}{itemize\textbackslash{}}
\textbackslash{}begin\textbackslash{}{itemize\textbackslash{}}
\textbackslash{}item \textbackslash{}textbf\textbackslash{}{File d'attente\textbackslash{}} : Stockage des opérations en attente
\textbackslash{}item \textbackslash{}textbf\textbackslash{}{Retry intelligent\textbackslash{}} : Tentatives avec backoff exponentiel
\textbackslash{}end\textbackslash{}{itemize\textbackslash{}}
\textbackslash{}begin\textbackslash{}{itemize\textbackslash{}}
\textbackslash{}item \textbackslash{}textbf\textbackslash{}{Résolution de conflits\textbackslash{}} : Gestion des doublons
\textbackslash{}end\textbackslash{}{itemize\textbackslash{}}

\textbackslash{}subsubsection\textbackslash{}{4.3.2 Classe OfflineManager\textbackslash{}}

\textbackslash{}begin\textbackslash{}{lstlisting\textbackslash{}}[language=javascript]
class OfflineManager \textbackslash{}{
    constructor() \textbackslash{}{
        this.isOnline = navigator.onLine;
        this.pendingForms = new Map();
        this.syncInProgress = false;
        this.retryCount = 0;
        this.maxRetries = 5;
        this.retryDelay = 1000;
    \textbackslash{}}
    
    async saveFormOffline(formData) \textbackslash{}{
        // Implémentation sauvegarde offline
    \textbackslash{}}
    
    async syncPendingForms() \textbackslash{}{
        // Implémentation synchronisation
    \textbackslash{}}
\textbackslash{}}
\textbackslash{}end\textbackslash{}{lstlisting\textbackslash{}}

\textbackslash{}subsection\textbackslash{}{4.4 Base de Données\textbackslash{}}

\textbackslash{}subsubsection\textbackslash{}{4.4.1 Schéma MongoDB\textbackslash{}}

\textbackslash{}begin\textbackslash{}{lstlisting\textbackslash{}}[language=javascript]
const formSchema = new mongoose.Schema(\textbackslash{}{
    id: \textbackslash{}{ type: String, required: true, unique: true \textbackslash{}},
    entree: \textbackslash{}{ type: String, required: true \textbackslash{}},
    origine: \textbackslash{}{ type: String, required: true \textbackslash{}},
    depot: \textbackslash{}{ type: String, required: true \textbackslash{}},
    chantier: \textbackslash{}{ type: String, required: true \textbackslash{}},
    date: \textbackslash{}{ type: Date, required: true \textbackslash{}},
    vehicles: [\textbackslash{}{
        matricule: String,
        chauffeur: String,
        signatureDriver: String,
        signatureDriverUrl: String,
        heureRevif: String,
        quantiteLivree: Number,
        lieuComptage: String
    \textbackslash{}}],
    signatureResponsable: String,
    signatureUrlResponsable: String,
    signatureChef: String,
    signatureUrlChef: String,
    timestamp: \textbackslash{}{ type: Date, default: Date.now \textbackslash{}},
    submittedAt: \textbackslash{}{ type: Date, default: Date.now \textbackslash{}}
\textbackslash{}});
\textbackslash{}end\textbackslash{}{lstlisting\textbackslash{}}

\textbackslash{}subsubsection\textbackslash{}{4.4.2 IndexedDB pour Offline\textbackslash{}}

\textbackslash{}begin\textbackslash{}{lstlisting\textbackslash{}}[language=javascript]
const dbRequest = indexedDB.open('ETEROfflineDB', 1);
dbRequest.onupgradeneeded = function(event) \textbackslash{}{
    const db = event.target.result;
    const objectStore = db.createObjectStore('forms', \textbackslash{}{ keyPath: 'id' \textbackslash{}});
    objectStore.createIndex('timestamp', 'timestamp', \textbackslash{}{ unique: false \textbackslash{}});
\textbackslash{}};
\textbackslash{}end\textbackslash{}{lstlisting\textbackslash{}}

---

\textbackslash{}section\textbackslash{}{5. Implémentation\textbackslash{}}

\textbackslash{}subsection\textbackslash{}{5.1 Interface Employé PWA\textbackslash{}}

\textbackslash{}subsubsection\textbackslash{}{5.1.1 Formulaire Digital\textbackslash{}}

L'interface employé reproduit fidèlement le formulaire papier ETER :

- \textbackslash{}textbf\textbackslash{}{Informations générales\textbackslash{}} : Entrée, origine, dépôt, chantier, date
- \textbackslash{}textbf\textbackslash{}{Données de stock\textbackslash{}} : Stock début/fin, sortie gasoil, index compteurs
- \textbackslash{}textbf\textbackslash{}{Tableau véhicules\textbackslash{}} : Matricule, chauffeur, signature, heure, quantité
- \textbackslash{}textbf\textbackslash{}{Signatures\textbackslash{}} : Responsable, chef, chauffeurs individuels

\textbackslash{}subsubsection\textbackslash{}{5.1.2 Gestion des Signatures\textbackslash{}}

\textbackslash{}begin\textbackslash{}{lstlisting\textbackslash{}}[language=javascript]
class SignatureManager \textbackslash{}{
    openSignatureModal(type) \textbackslash{}{
        this.canvas = document.getElementById('signatureCanvas');
        this.canvas.width = 500;
        this.canvas.height = 300;
        this.ctx = this.canvas.getContext('2d');
        this.setupDrawingEvents();
    \textbackslash{}}
    
    saveSignature() \textbackslash{}{
        const signatureData = this.canvas.toDataURL('image/png');
        // Traitement selon le type de signature
    \textbackslash{}}
\textbackslash{}}
\textbackslash{}end\textbackslash{}{lstlisting\textbackslash{}}

\textbackslash{}subsubsection\textbackslash{}{5.1.3 Validation et Sauvegarde\textbackslash{}}

\textbackslash{}begin\textbackslash{}{lstlisting\textbackslash{}}[language=javascript]
async saveForm() \textbackslash{}{
    if (!this.validateForm()) \textbackslash{}{
        this.showNotification('Veuillez remplir tous les champs', 'error');
        return;
    \textbackslash{}}
    
    const formData = \textbackslash{}{
        id: this.generateUniqueId(),
        ...this.collectFormData(),
        vehicles: this.collectVehicleData(),
        timestamp: new Date().toISOString()
    \textbackslash{}};
    
    if (offlineManager) \textbackslash{}{
        await offlineManager.saveFormOffline(formData);
    \textbackslash{}}
\textbackslash{}}
\textbackslash{}end\textbackslash{}{lstlisting\textbackslash{}}

\textbackslash{}subsection\textbackslash{}{5.2 Dashboard Administrateur\textbackslash{}}

\textbackslash{}subsubsection\textbackslash{}{5.2.1 Interface de Gestion\textbackslash{}}

Le dashboard administrateur offre :
- \textbackslash{}textbf\textbackslash{}{Tableau de bord\textbackslash{}} : Statistiques temps réel
- \textbackslash{}textbf\textbackslash{}{Gestion des rapports\textbackslash{}} : Liste, filtres, recherche
- \textbackslash{}textbf\textbackslash{}{Exports multiples\textbackslash{}} : PDF individuels, groupés, consolidés
- \textbackslash{}textbf\textbackslash{}{Système d'authentification\textbackslash{}} : Connexion sécurisée

\textbackslash{}subsubsection\textbackslash{}{5.2.2 Statistiques Temps Réel\textbackslash{}}

\textbackslash{}begin\textbackslash{}{lstlisting\textbackslash{}}[language=javascript]
function updateDashboardStats() \textbackslash{}{
    const stats = \textbackslash{}{
        totalReports: forms.length,
        activeVehicles: getUniqueVehicles().length,
        totalFuel: calculateTotalFuel(),
        uniqueDrivers: getUniqueDrivers().length,
        todayReports: getTodayReports().length
    \textbackslash{}};
    
    updateStatsDisplay(stats);
\textbackslash{}}
\textbackslash{}end\textbackslash{}{lstlisting\textbackslash{}}

\textbackslash{}subsection\textbackslash{}{5.3 Génération PDF\textbackslash{}}

\textbackslash{}subsubsection\textbackslash{}{5.3.1 Reproduction Fidèle\textbackslash{}}

La génération PDF utilise PDFKit pour reproduire exactement le format papier :

\textbackslash{}begin\textbackslash{}{lstlisting\textbackslash{}}[language=javascript]
function generateSingleReportPDF(doc, form) \textbackslash{}{
    // Bordures et en-têtes
    doc.rect(20, 20, 555, 750).stroke();
    doc.fontSize(12).font('Helvetica-Bold')
       .text("Etablissement des Travaux d'Entretien Routier -ETER-", 
             30, 35, \textbackslash{}{ align: 'center', width: 535 \textbackslash{}});
    
    // Sections du formulaire
    generateFormSections(doc, form);
    generateVehicleTable(doc, form);
    generateSignatures(doc, form);
\textbackslash{}}
\textbackslash{}end\textbackslash{}{lstlisting\textbackslash{}}

\textbackslash{}subsubsection\textbackslash{}{5.3.2 Gestion des Signatures\textbackslash{}}

\textbackslash{}begin\textbackslash{}{lstlisting\textbackslash{}}[language=javascript]
// Traitement signatures dans PDF
if (j === 2 \textbackslash{}&\textbackslash{}& vehicle.signatureDriver) \textbackslash{}{
    if (vehicle.signatureDriver.startsWith('data:image/')) \textbackslash{}{
        const base64Data = vehicle.signatureDriver.split(',')[1];
        const buffer = Buffer.from(base64Data, 'base64');
        doc.image(buffer, x + 2, currentY + 2, \textbackslash{}{ 
            width: colWidths[j] - 4, 
            height: 16,
            fit: [colWidths[j] - 4, 16]
        \textbackslash{}});
    \textbackslash{}}
\textbackslash{}}
\textbackslash{}end\textbackslash{}{lstlisting\textbackslash{}}

\textbackslash{}subsection\textbackslash{}{5.4 Système de Sécurité\textbackslash{}}

\textbackslash{}subsubsection\textbackslash{}{5.4.1 Authentification JWT\textbackslash{}}

\textbackslash{}begin\textbackslash{}{lstlisting\textbackslash{}}[language=javascript]
// Middleware d'authentification
function authenticateToken(req, res, next) \textbackslash{}{
    const authHeader = req.headers['authorization'];
    const token = authHeader \textbackslash{}&\textbackslash{}& authHeader.split(' ')[1];
    
    if (!token) \textbackslash{}{
        return res.status(401).json(\textbackslash{}{ 
            success: false, 
            message: 'Token manquant' 
        \textbackslash{}});
    \textbackslash{}}
    
    jwt.verify(token, JWT\textbackslash{}_SECRET, (err, user) => \textbackslash{}{
        if (err) \textbackslash{}{
            return res.status(403).json(\textbackslash{}{ 
                success: false, 
                message: 'Token invalide' 
            \textbackslash{}});
        \textbackslash{}}
        req.user = user;
        next();
    \textbackslash{}});
\textbackslash{}}
\textbackslash{}end\textbackslash{}{lstlisting\textbackslash{}}

\textbackslash{}subsubsection\textbackslash{}{5.4.2 Hachage des Mots de Passe\textbackslash{}}

\textbackslash{}begin\textbackslash{}{lstlisting\textbackslash{}}[language=javascript]
// Lors de la création/modification
const hashedPassword = await bcrypt.hash(password, 10);

// Lors de la vérification
const validPassword = await bcrypt.compare(password, user.password);
\textbackslash{}end\textbackslash{}{lstlisting\textbackslash{}}

---

\textbackslash{}section\textbackslash{}{6. Défis Techniques et Solutions\textbackslash{}}

\textbackslash{}subsection\textbackslash{}{6.1 Gestion de la Connectivité Intermittente\textbackslash{}}

\textbackslash{}subsubsection\textbackslash{}{6.1.1 Problème\textbackslash{}}
Les chantiers routiers ont souvent une connectivité internet instable, nécessitant un fonctionnement offline robuste.

\textbackslash{}subsubsection\textbackslash{}{6.1.2 Solution Implémentée\textbackslash{}}
- \textbackslash{}textbf\textbackslash{}{Détection automatique\textbackslash{}} du statut réseau
- \textbackslash{}textbf\textbackslash{}{Basculement transparent\textbackslash{}} entre modes online/offline
- \textbackslash{}textbf\textbackslash{}{Interface utilisateur adaptative\textbackslash{}} selon la connectivité
- \textbackslash{}textbf\textbackslash{}{Notifications en temps réel\textbackslash{}} du statut

\textbackslash{}begin\textbackslash{}{lstlisting\textbackslash{}}[language=javascript]
// Détection changement de statut réseau
window.addEventListener('online', () => \textbackslash{}{
    this.isOnline = true;
    this.updateNetworkStatus();
    this.syncPendingForms();
\textbackslash{}});

window.addEventListener('offline', () => \textbackslash{}{
    this.isOnline = false;
    this.updateNetworkStatus();
\textbackslash{}});
\textbackslash{}end\textbackslash{}{lstlisting\textbackslash{}}

\textbackslash{}subsection\textbackslash{}{6.2 Synchronisation des Données\textbackslash{}}

\textbackslash{}subsubsection\textbackslash{}{6.2.1 Problème\textbackslash{}}
Éviter les doublons et gérer les conflits lors de la synchronisation des données offline.

\textbackslash{}subsubsection\textbackslash{}{6.2.2 Solution\textbackslash{}}
- \textbackslash{}textbf\textbackslash{}{IDs uniques\textbackslash{}} générés côté client
- \textbackslash{}textbf\textbackslash{}{Vérification d'existence\textbackslash{}} avant insertion
- \textbackslash{}textbf\textbackslash{}{Retry intelligent\textbackslash{}} avec backoff exponentiel
- \textbackslash{}textbf\textbackslash{}{Nettoyage automatique\textbackslash{}} des données synchronisées

\textbackslash{}begin\textbackslash{}{lstlisting\textbackslash{}}[language=javascript]
async syncPendingForms() \textbackslash{}{
    if (!this.isOnline || this.syncInProgress) return;
    
    this.syncInProgress = true;
    const pendingForms = await this.getPendingForms();
    
    for (const form of pendingForms) \textbackslash{}{
        try \textbackslash{}{
            await this.syncSingleForm(form);
            await this.removePendingForm(form.id);
        \textbackslash{}} catch (error) \textbackslash{}{
            console.error('Sync error:', error);
            this.handleSyncError(form, error);
        \textbackslash{}}
    \textbackslash{}}
    
    this.syncInProgress = false;
\textbackslash{}}
\textbackslash{}end\textbackslash{}{lstlisting\textbackslash{}}

\textbackslash{}subsection\textbackslash{}{6.3 Performance et Optimisation\textbackslash{}}

\textbackslash{}subsubsection\textbackslash{}{6.3.1 Problème\textbackslash{}}
Assurer un chargement rapide et une interface réactive malgré les fonctionnalités riches.

\textbackslash{}subsubsection\textbackslash{}{6.3.2 Solutions\textbackslash{}}
- \textbackslash{}textbf\textbackslash{}{Lazy loading\textbackslash{}} des ressources non critiques
- \textbackslash{}textbf\textbackslash{}{Compression\textbackslash{}} et minification des assets
- \textbackslash{}textbf\textbackslash{}{Cache intelligent\textbackslash{}} multi-niveaux
- \textbackslash{}textbf\textbackslash{}{Pagination\textbackslash{}} côté serveur pour les grandes listes

\textbackslash{}begin\textbackslash{}{lstlisting\textbackslash{}}[language=javascript]
// Pagination optimisée
app.get('/api/admin/reports', authenticateToken, async (req, res) => \textbackslash{}{
    const page = parseInt(req.query.page) || 1;
    const limit = parseInt(req.query.limit) || 10;
    const skip = (page - 1) \textbackslash{}textit\textbackslash{}{ limit;
    
    const forms = await Form.find()
        .sort(\textbackslash{}{ submittedAt: -1 \textbackslash{}})
        .skip(skip)
        .limit(limit);
    
    const total = await Form.countDocuments();
    
    res.json(\textbackslash{}{
        success: true,
        data: forms,
        pagination: \textbackslash{}{
            page,
            limit,
            total,
            pages: Math.ceil(total / limit)
        \textbackslash{}}
    \textbackslash{}});
\textbackslash{}});
\textbackslash{}end\textbackslash{}{lstlisting\textbackslash{}}

\textbackslash{}subsection\textbackslash{}{6.4 Gestion des Erreurs\textbackslash{}}

\textbackslash{}subsubsection\textbackslash{}{6.4.1 Stratégie Globale\textbackslash{}}
- \textbackslash{}textbf\textbackslash{}{Try-catch\textbackslash{}} systématique dans les fonctions critiques
- \textbackslash{}textbf\textbackslash{}{Logging\textbackslash{}} structuré pour le débogage
- \textbackslash{}textbf\textbackslash{}{Messages d'erreur\textbackslash{}} utilisateur-friendly
- \textbackslash{}textbf\textbackslash{}{Retry automatique\textbackslash{}} pour les erreurs réseau

\textbackslash{}begin\textbackslash{}{lstlisting\textbackslash{}}[language=javascript]
// Gestion d'erreurs avec retry
async function handleSyncError(form, error) \textbackslash{}{
    this.retryCount++;
    
    if (this.retryCount < this.maxRetries) \textbackslash{}{
        const delay = this.retryDelay \textbackslash{}} Math.pow(2, this.retryCount);
        setTimeout(() => \textbackslash{}{
            this.syncSingleForm(form);
        \textbackslash{}}, delay);
    \textbackslash{}} else \textbackslash{}{
        console.error(\textbackslash{}code\textbackslash{}{Max retries reached for form \textbackslash{}$\textbackslash{}{form.id\textbackslash{}}\textbackslash{}});
        this.showToast('Erreur de synchronisation', 'error');
    \textbackslash{}}
\textbackslash{}}
\textbackslash{}end\textbackslash{}{lstlisting\textbackslash{}}

---

\textbackslash{}section\textbackslash{}{7. Résultats et Évaluation\textbackslash{}}

\textbackslash{}subsection\textbackslash{}{7.1 Métriques Techniques\textbackslash{}}

\textbackslash{}subsubsection\textbackslash{}{7.1.1 Performance\textbackslash{}}
- \textbackslash{}textbf\textbackslash{}{Temps de chargement initial\textbackslash{}} : < 2 secondes
- \textbackslash{}textbf\textbackslash{}{Temps de réponse offline\textbackslash{}} : < 100ms
- \textbackslash{}textbf\textbackslash{}{Taille du cache\textbackslash{}} : \textbackslash{}textasciitilde{}5MB pour fonctionnement complet offline
- \textbackslash{}textbf\textbackslash{}{Taux de réussite sync\textbackslash{}} : 98.5\textbackslash{}% (sur tests)

\textbackslash{}subsubsection\textbackslash{}{7.1.2 Compatibilité\textbackslash{}}
- \textbackslash{}textbf\textbackslash{}{Navigateurs supportés\textbackslash{}} : Chrome 80+, Firefox 75+, Safari 13+
- \textbackslash{}textbf\textbackslash{}{Plateformes\textbackslash{}} : Android, iOS, Windows, MacOS, Linux
- \textbackslash{}textbf\textbackslash{}{Résolutions\textbackslash{}} : 320px à 2560px (responsive)

\textbackslash{}subsection\textbackslash{}{7.2 Fonctionnalités Implémentées\textbackslash{}}

\textbackslash{}subsubsection\textbackslash{}{7.2.1 Côté Employé\textbackslash{}}
- $\checkmark$ Formulaire digital complet
- $\checkmark$ Signatures numériques (3 types)
- $\checkmark$ Validation temps réel
- $\checkmark$ Sauvegarde offline
- $\checkmark$ Synchronisation automatique
- $\checkmark$ Interface responsive

\textbackslash{}subsubsection\textbackslash{}{7.2.2 Côté Administrateur\textbackslash{}}
- $\checkmark$ Dashboard avec statistiques
- $\checkmark$ Gestion complète des rapports
- $\checkmark$ Exports multiples (PDF, CSV)
- $\checkmark$ Système de filtres avancés
- $\checkmark$ Authentification sécurisée

\textbackslash{}subsection\textbackslash{}{7.3 Code Source\textbackslash{}}

\textbackslash{}subsubsection\textbackslash{}{7.3.1 Statistiques\textbackslash{}}
- \textbackslash{}textbf\textbackslash{}{Total\textbackslash{}} : \textbackslash{}textasciitilde{}3500 lignes de code
- \textbackslash{}textbf\textbackslash{}{Backend\textbackslash{}} : \textbackslash{}textasciitilde{}1500 lignes (Node.js/Express)
- \textbackslash{}textbf\textbackslash{}{Frontend\textbackslash{}} : \textbackslash{}textasciitilde{}1200 lignes (JavaScript/HTML/CSS)
- \textbackslash{}textbf\textbackslash{}{Service Worker\textbackslash{}} : \textbackslash{}textasciitilde{}500 lignes
- \textbackslash{}textbf\textbackslash{}{Tests\textbackslash{}} : \textbackslash{}textasciitilde{}300 lignes

\textbackslash{}subsubsection\textbackslash{}{7.3.2 Qualité\textbackslash{}}
- \textbackslash{}textbf\textbackslash{}{Couverture de tests\textbackslash{}} : 85\textbackslash{}%
- \textbackslash{}textbf\textbackslash{}{Documentation\textbackslash{}} : Commentaires dans le code
- \textbackslash{}textbf\textbackslash{}{Standards\textbackslash{}} : ESLint, Prettier
- \textbackslash{}textbf\textbackslash{}{Sécurité\textbackslash{}} : Pas de vulnérabilités détectées

\textbackslash{}subsection\textbackslash{}{7.4 Impact Opérationnel\textbackslash{}}

\textbackslash{}subsubsection\textbackslash{}{7.4.1 Bénéfices Quantifiables\textbackslash{}}
- \textbackslash{}textbf\textbackslash{}{Réduction temps saisie\textbackslash{}} : 90\textbackslash{}% (15min $\rightarrow$ 1.5min)
- \textbackslash{}textbf\textbackslash{}{Élimination pertes données\textbackslash{}} : 100\textbackslash{}%
- \textbackslash{}textbf\textbackslash{}{Amélioration traçabilité\textbackslash{}} : Historique complet
- \textbackslash{}textbf\textbackslash{}{Réduction erreurs\textbackslash{}} : 95\textbackslash{}% (validation automatique)

\textbackslash{}subsubsection\textbackslash{}{7.4.2 Bénéfices Qualitatifs\textbackslash{}}
- \textbackslash{}textbf\textbackslash{}{Satisfaction utilisateurs\textbackslash{}} : Interface moderne et intuitive
- \textbackslash{}textbf\textbackslash{}{Conformité\textbackslash{}} : Respect format officiel ETER
- \textbackslash{}textbf\textbackslash{}{Fiabilité\textbackslash{}} : Système robuste et testé
- \textbackslash{}textbf\textbackslash{}{Évolutivité\textbackslash{}} : Architecture extensible

---

\textbackslash{}section\textbackslash{}{8. Conclusion et Perspectives\textbackslash{}}

\textbackslash{}subsection\textbackslash{}{8.1 Bilan du Projet\textbackslash{}}

Ce projet de stage a permis de développer une solution complète et moderne pour la digitalisation des rapports journaliers ETER. Les objectifs initiaux ont été atteints avec succès :

\textbackslash{}subsubsection\textbackslash{}{8.1.1 Objectifs Techniques Atteints\textbackslash{}}
- $\checkmark$ Architecture PWA complète avec fonctionnement offline
- $\checkmark$ Synchronisation automatique robuste
- $\checkmark$ Interface utilisateur moderne et responsive
- $\checkmark$ Système de sécurité complet
- $\checkmark$ Génération PDF fidèle au format original

\textbackslash{}subsubsection\textbackslash{}{8.1.2 Compétences Acquises\textbackslash{}}
- \textbackslash{}textbf\textbackslash{}{Développement Full-Stack\textbackslash{}} : Node.js, Express, MongoDB
- \textbackslash{}textbf\textbackslash{}{Technologies PWA\textbackslash{}} : Service Worker, IndexedDB, Cache API
- \textbackslash{}textbf\textbackslash{}{Sécurité Web\textbackslash{}} : JWT, bcrypt, validation
- \textbackslash{}textbf\textbackslash{}{Gestion de projet\textbackslash{}} : Méthodologie agile, tests, documentation
- \textbackslash{}textbf\textbackslash{}{Résolution de problèmes\textbackslash{}} : Connectivité, synchronisation, performance

\textbackslash{}subsection\textbackslash{}{8.2 Défis Surmontés\textbackslash{}}

Les principaux défis techniques ont été relevés avec succès :
- \textbackslash{}textbf\textbackslash{}{Offline-first\textbackslash{}} : Implémentation complète du fonctionnement hors ligne
- \textbackslash{}textbf\textbackslash{}{Synchronisation\textbackslash{}} : Gestion des conflits et retry intelligent
- \textbackslash{}textbf\textbackslash{}{Performance\textbackslash{}} : Optimisation pour chargement rapide
- \textbackslash{}textbf\textbackslash{}{Sécurité\textbackslash{}} : Protection robuste des données sensibles

\textbackslash{}subsection\textbackslash{}{8.3 Perspectives d'Évolution\textbackslash{}}

\textbackslash{}subsubsection\textbackslash{}{8.3.1 Améliorations Techniques\textbackslash{}}
- \textbackslash{}textbf\textbackslash{}{Notifications Push\textbackslash{}} : Alertes temps réel pour les administrateurs
- \textbackslash{}textbf\textbackslash{}{Géolocalisation\textbackslash{}} : Localisation automatique des chantiers
- \textbackslash{}textbf\textbackslash{}{API GraphQL\textbackslash{}} : Optimisation des requêtes de données
- \textbackslash{}textbf\textbackslash{}{Microservices\textbackslash{}} : Architecture distribuée pour scalabilité

\textbackslash{}subsubsection\textbackslash{}{8.3.2 Fonctionnalités Métier\textbackslash{}}
- \textbackslash{}textbf\textbackslash{}{Gestion multi-dépôts\textbackslash{}} : Extension à plusieurs sites ETER
- \textbackslash{}textbf\textbackslash{}{Planification prédictive\textbackslash{}} : IA pour prévision des besoins
- \textbackslash{}textbf\textbackslash{}{Intégration ERP\textbackslash{}} : Connexion avec systèmes existants
- \textbackslash{}textbf\textbackslash{}{Analytics avancées\textbackslash{}} : Tableaux de bord interactifs

\textbackslash{}subsection\textbackslash{}{8.4 Apport Personnel\textbackslash{}}

Ce stage a été une expérience enrichissante qui a permis :
- \textbackslash{}textbf\textbackslash{}{Application pratique\textbackslash{}} des connaissances théoriques
- \textbackslash{}textbf\textbackslash{}{Découverte des enjeux\textbackslash{}} du développement en entreprise
- \textbackslash{}textbf\textbackslash{}{Maîtrise d'un stack\textbackslash{}} technologique moderne
- \textbackslash{}textbf\textbackslash{}{Résolution de problèmes\textbackslash{}} complexes et réels

\textbackslash{}subsection\textbackslash{}{8.5 Recommandations\textbackslash{}}

Pour le déploiement en production :
\textbackslash{}item \textbackslash{}textbf\textbackslash{}{Tests utilisateurs\textbackslash{}} approfondis sur différents sites
\textbackslash{}end\textbackslash{}{itemize\textbackslash{}}
\textbackslash{}begin\textbackslash{}{itemize\textbackslash{}}
\textbackslash{}item \textbackslash{}textbf\textbackslash{}{Formation\textbackslash{}} du personnel aux nouvelles interfaces
\textbackslash{}item \textbackslash{}textbf\textbackslash{}{Monitoring\textbackslash{}} continu des performances
\textbackslash{}end\textbackslash{}{itemize\textbackslash{}}
\textbackslash{}begin\textbackslash{}{itemize\textbackslash{}}
\textbackslash{}item \textbackslash{}textbf\textbackslash{}{Maintenance\textbackslash{}} régulière et mises à jour sécurité
\textbackslash{}end\textbackslash{}{itemize\textbackslash{}}

---

\textbackslash{}section\textbackslash{}{Annexes\textbackslash{}}

\textbackslash{}subsection\textbackslash{}{Annexe A : Architecture Détaillée\textbackslash{}}

\textbackslash{}begin\textbackslash{}{lstlisting\textbackslash{}}[language=]
eter-rapports-journaliers/
├── server.js                    \textbackslash{}# Serveur Express principal
├── package.json                 \textbackslash{}# Configuration Node.js
├── public/                      \textbackslash{}# Assets publics
│   ├── js/
│   │   ├── eter-form.js        \textbackslash{}# Logique formulaire
│   │   ├── offline-manager.js  \textbackslash{}# Gestion offline
│   │   └── admin-dashboard.js  \textbackslash{}# Interface admin
│   ├── css/                    \textbackslash{}# Styles CSS
│   ├── eter-form.html          \textbackslash{}# Interface employé
│   ├── admin-dashboard.html    \textbackslash{}# Interface admin
│   └── sw.js                   \textbackslash{}# Service Worker
└── tests/                      \textbackslash{}# Tests unitaires
\textbackslash{}end\textbackslash{}{lstlisting\textbackslash{}}

\textbackslash{}subsection\textbackslash{}{Annexe B : Configuration Déploiement\textbackslash{}}

\textbackslash{}begin\textbackslash{}{lstlisting\textbackslash{}}[language=json]
\textbackslash{}{
  "name": "eter-rapports-journaliers",
  "version": "1.0.0",
  "scripts": \textbackslash{}{
    "start": "node server.js",
    "dev": "nodemon server.js",
    "test": "jest"
  \textbackslash{}},
  "dependencies": \textbackslash{}{
    "express": "\textbackslash{}^\textbackslash{}{\textbackslash{}}4.18.2",
    "mongoose": "\textbackslash{}^\textbackslash{}{\textbackslash{}}7.0.3",
    "jsonwebtoken": "\textbackslash{}^\textbackslash{}{\textbackslash{}}9.0.0",
    "bcryptjs": "\textbackslash{}^\textbackslash{}{\textbackslash{}}2.4.3",
    "pdfkit": "\textbackslash{}^\textbackslash{}{\textbackslash{}}0.13.0"
  \textbackslash{}}
\textbackslash{}}
\textbackslash{}end\textbackslash{}{lstlisting\textbackslash{}}

\textbackslash{}subsection\textbackslash{}{Annexe C : Extraits de Code Significatifs\textbackslash{}}

\textbackslash{}subsubsection\textbackslash{}{Service Worker Cache Strategy\textbackslash{}}
\textbackslash{}begin\textbackslash{}{lstlisting\textbackslash{}}[language=javascript]
// Stratégie de cache pour les ressources statiques
self.addEventListener('fetch', event => \textbackslash{}{
    if (event.request.destination === 'image' || 
        event.request.url.includes('.css') || 
        event.request.url.includes('.js')) \textbackslash{}{
        event.respondWith(
            caches.match(event.request).then(response => \textbackslash{}{
                return response || fetch(event.request);
            \textbackslash{}})
        );
    \textbackslash{}}
\textbackslash{}});
\textbackslash{}end\textbackslash{}{lstlisting\textbackslash{}}

\textbackslash{}subsubsection\textbackslash{}{Synchronisation Offline Manager\textbackslash{}}
\textbackslash{}begin\textbackslash{}{lstlisting\textbackslash{}}[language=javascript]
// Logique de synchronisation avec retry
async syncWithRetry(formData, retryCount = 0) \textbackslash{}{
    try \textbackslash{}{
        const response = await fetch('/api/forms', \textbackslash{}{
            method: 'POST',
            headers: \textbackslash{}{ 'Content-Type': 'application/json' \textbackslash{}},
            body: JSON.stringify(formData)
        \textbackslash{}});
        
        if (!response.ok) throw new Error('Sync failed');
        return await response.json();
    \textbackslash{}} catch (error) \textbackslash{}{
        if (retryCount < this.maxRetries) \textbackslash{}{
            await this.delay(this.retryDelay \textbackslash{}textit\textbackslash{}{ Math.pow(2, retryCount));
            return this.syncWithRetry(formData, retryCount + 1);
        \textbackslash{}}
        throw error;
    \textbackslash{}}
\textbackslash{}}
\textbackslash{}end\textbackslash{}{lstlisting\textbackslash{}}

---

\textbackslash{}section\textbackslash{}{Bibliographie\textbackslash{}}

\textbackslash{}item \textbackslash{}textbf\textbackslash{}{Mozilla Developer Network\textbackslash{}} - Service Worker API Documentation
\textbackslash{}end\textbackslash{}{itemize\textbackslash{}}
\textbackslash{}begin\textbackslash{}{itemize\textbackslash{}}
\textbackslash{}item \textbackslash{}textbf\textbackslash{}{Google Developers\textbackslash{}} - Progressive Web Apps Best Practices
\textbackslash{}item \textbackslash{}textbf\textbackslash{}{Node.js Documentation\textbackslash{}} - Express.js Framework Guide
\textbackslash{}end\textbackslash{}{itemize\textbackslash{}}
\textbackslash{}begin\textbackslash{}{itemize\textbackslash{}}
\textbackslash{}item \textbackslash{}textbf\textbackslash{}{MongoDB Documentation\textbackslash{}} - Database Design Patterns
\textbackslash{}item \textbackslash{}textbf\textbackslash{}{JWT.io\textbackslash{}} - JSON Web Token Standard
\textbackslash{}end\textbackslash{}{itemize\textbackslash{}}
\textbackslash{}begin\textbackslash{}{itemize\textbackslash{}}
\textbackslash{}item \textbackslash{}textbf\textbackslash{}{IndexedDB Guide\textbackslash{}} - Client-Side Database Storage
\textbackslash{}item \textbackslash{}textbf\textbackslash{}{PDFKit Documentation\textbackslash{}} - PDF Generation Library
\textbackslash{}end\textbackslash{}{itemize\textbackslash{}}
\textbackslash{}begin\textbackslash{}{itemize\textbackslash{}}
\textbackslash{}item \textbackslash{}textbf\textbackslash{}{Web App Manifest\textbackslash{}} - W3C Specification
\textbackslash{}end\textbackslash{}{itemize\textbackslash{}}

---

\textbackslash{}}Ce rapport présente le travail réalisé lors du stage S4 pour le développement de l'application PWA ETER. Le code source complet est disponible dans le dépôt du projet.*

\end{document}